\documentclass[a4paper]{jpconf}
\usepackage{graphicx}

\begin{document}
\title{Vibrational characteristics of a superconducting magnetic bearing employed for a prototype polarization modulator}

\author{Yuki Sakurai$^{1}$, Tomotake Matsumura$^{1}$, Hajime Sugai$^{1}$, Nobuhiko Katayama$^{1}$, Hiroyuki Ohsaki$^{2}$, Yutaka Terao$^{2}$, Yusuke Terachi$^{2}$, Hirokazu Kataza$^{3}$, Shin Utsunomiya$^{1}$, Ryo Yamamoto$^{3}$}
\vspace{2mm}
\address{
$^{1}$Kavli Institute for the Physics and Mathematics of the Universe (WPI),The University of Tokyo Institutes for Advanced Study, The University of Tokyo, 5-1-5 Kashiwanoha, Kashiwa, Chiba 277-8583, Japan \\
$^{2}$Graduate School of Frontier Sciences, The University of Tokyo, 5-1-5 Kashiwanoha, Kashiwa, Chiba 277-8561, Japan \\
$^{3}$Japan Aerospace Exploration Agency, Institute of Space and Astronautical Science (ISAS), 3-1-1 Yoshinodai, Chuo-ku, Sagamihara, Kanagawa 252-5210, Japan
}

\ead{yuki.sakurai@ipmu.jp, tomotake.matsumura@ipmu.jp}

\begin{abstract}
We present the vibrational characteristics of a levitating rotor in a superconducting magnetic bearing (SMB) system operating at below 10 K.
We develop a polarization modulator that requires a continuously rotating optical element, called half-wave plate (HWP), for a cosmic microwave background polarization experiment.
The HWP has to operate at the temperature below 10 K, and thus an SMB provides a smooth rotation of the HWP at the cryogenic temperature at about 10K with minimal heat dissipation.
In order to understand the potential interference to the cosmological observations due to the vibration of the HWP,
it is essential to characterize the vibrational properties of the levitating rotor of the SMB.
We constructed a prototype model that consists of an SMB with an array of high temperature superconductors, YBCO, and a permanent magnet ring.
The rotor position is monitored by the laser displacement gage, and the cryogenic Hall sensor via the magnetic field.
In this presentation, we present the measurement results of the vibration characteristics using our prototype SMB system.
We discuss the spring constant and the magnetic field inhomogeneity of the SMB and sub-components including a rotation frequency monitoring system, a holder mechanism, and drive system.
Finally, we discuss the projected performance of this technology toward the use in future space missions.
\end{abstract}

\section*{Introduction}

\begin{figure}[bh]
   \centering
   \includegraphics[width=70mm]{D400mm.eps}
   \caption{The experimental setup for the spin down measurements for $D=384$~mm}
   \label{fig:d400}
\end{figure}

\section*{Acknowledgment}
The author would like to thank to Dr. H. Imada at ISAS/JAXA.
This work was supported by MEXT KAKENHI Grant Numbers JP15H05441 and JSPS Core-to-Core Program, A. Advanced Research Networks.
This work was also supported by World Premier International Research Center Initiative (WPI), MEXT, Japan.


\end{document}


